\noindent Articles in students journal \macierzator (ISSN 2083-9774)\footnote{in square brackets the translation of the title is given}:
\begin{itemize}
  \item \textsl{Kilka ciekawostek o spirali logarytmicznej} [\textsl{A few interesting facts about logarithmic spiral}] (Macierzator22, April 2009)
  \item \textsl{Liczby pierwsze na płaszczyźnie} [\textsl{Primes on the plane}] (Macierzator23, May 2009)
  \item \textsl{Kiedy sumy cyfr rozwinięć liczb $\pi$ i $e$ są takie same?} [\textsl{When the sum of the digits of decimal representation of the $\pi$ and $e$ are the same?}] (Macierzator24, October 2009)
  \item \textsl{Funkcja Ackermanna} [\textsl{Ackermann function}] (Macierzator25, November 2009)
  \item \textsl{O kilku strasznych dowodach} [\textsl{About few scary proofs}] (Macierzator26, December 2009)
  \item \textsl{O samopodobieństwie  trochę inaczej} [\textsl{About self-similarity in different way}] (Macierzator27, January 2010)
  \item \textsl{O pewnej maszynie} [\textsl{About some machine}] (Macierzator28, February 2010)
  \item \textsl{Gonitwa na płaszczyźnie} [\textsl{Chase on the plane}] (Macierzator33, November 2010)
  \item \textsl{O tym, że matematyka potrafi być czasami bardzo złośliwa} [\textsl{Mathematics can be sometimes very nasty}] (Macierzator34, December 2010)
  \item \textsl{O pewnej granicy, rozwinięciu binarnym i logarytmie naturalnym} [\textsl{About some limit, binary expansion and natural logarithm}] (Macierzator35, February 2011)
  \item \textsl{Twierdzenie Midy'ego} [\textsl{Midy's theorem}] ($\pi$-Macierzator36, March 2011)
  \item \textsl{Twierdzenia Tarskiego o punkcie stałym} [\textsl{Tarski's fixed point theorem}] (co-authors: Marek Biedrzycki, Szymon Draga, Mateusz Jurczyński, Jolanta Marzec, Weronika Siwek, Mikołaj Stańczyk, Aleksandra Urban, Peter Volkmann) (Macierzator38, May-June 2011)
  \item \textsl{Zasada szufladkowa - kilka prostych wniosków} [\textsl{Pigeonhole principle - few interesting corollaries}] (Macierzator39, September 2011)
  \item \textsl{Spowalnianie rozbieżności szeregów} [\textsl{Decreasing the rate of divergence of series}] (Macierzator41, November 2011)
  \item \textsl{O pewnym ciekawym szeregu} [\textsl{On some interesting series}] ($\pi$-Macierzator45, March 2012)
  \item \textsl{O gęstości obrazu liczb naturalnych poprzez funkcję sinus} [\textsl{On the density of the image of naturals by sine}] (Macierzator46, April 2012)
  \item \textsl{Przykład Dieudonné'a równania różniczkowego bez rozwiązania} [\textsl{Dieudonné's example of differential equation without solution}] (co-authors: Wojciech Bielas, Szymon Draga, Konrad Jałowiecki, Magdalena Nowak, Agnieszka Piszczek, Magdalena Sitko, Peter Volkmann, Adam Wrzesiński, Radosław Zawiski, Joanna Zwierzyńska) (Macierzator47, May 2012)
  \item \textsl{Matematyczne polowanie na zająca} [\textsl{Mathematical hunt for a hare}] ($\pi$-Macierzator 56, March 2014)
\end{itemize}
