\noindent Given talks\footnote{in square brackets the translation of the title is given}:
\begin{itemize}
  \item 31.05.2008 --- \textsl{Generatory liczb pseudolosowych} [\textsl{Pseudo-Random Number Generators}] (XXIV SMS' Session \textsl{Applications of mathematics}, Szczyrk)
  \item 10.10.2008 --- \textsl{Prymitywy matematyczne} [\textsl{Mathematical primitives}] (talk for high schools)
  \item 17.10.2008 --- \textsl{Iloczyny liczb pierwszych a hipersześciany} [\textsl{Prime number products and hypercubes}] (talk for members of the SMS)
  \item 29.11.2008 --- \textsl{O liczbach leptonowych i kwarkowych} [\textsl{Lepton and quark numbers}] (XXV SMS' Session \textsl{Numbers}, Szczyrk)
  \item 19.12.2008 --- \textsl{Konstrukcje geometryczne} [\textsl{Geometric constructions}] (talk for high schools)
  \item 06.03.2009 --- \textsl{Przestrzenie metryczne, czyli... jak matematyk mierzy odległość} [\textsl{Metric spaces --- mathematician's means to count the distance}] (Spotkanie z Królową Nauk, Chorzów)
  \item 13.03.2009 --- \textsl{Gra w życie} [\textsl{Life}] (The $\pi$ day)
  \item 19.03.2009 --- \textsl{Fraktale} [\textsl{Fractals}] (Scientific festival, Tychy)
  \item 03.04.2009 --- \textsl{Jak obliczyć $\sin x$?} [\textsl{How to evaluate $\sin x$?}] (talk for high schools)
  \item 01.05.2009 --- \textsl{O nieciągłych rozwiązaniach równania Cauchy'ego} [\textsl{Discontinuous solutions of the Cauchy equation}] (XXVI SMS' Session \textsl{Equations and Inequalities}, Szczyrk)
  \item 29.05.2009 --- \textsl{Kiedy $\sin x=2$?} [\textsl{When $\sin x=2$ holds?}] (talk for high schools)
  \item 06.07.2009 --- \textsl{Kilka słów o liczbach Liouvielle'a} [\textsl{Few words about Liouville numbers}] (I mathematical summer school of Cracow University of Technology, Krynica Górska)
  \item 01.09.2009 --- \textsl{Zbiory niemierzalne a równanie Cauchy'ego} [\textsl{Non-measurable sets and Cauchy functional equation}] (Summer Maths Workshop 2009, Toruń)
  \item 09.10.2009 --- \textsl{O rysowaniu wykresów funkcji} [\textsl{Plotting function graphs}] (talk for high schools)
  \item 07.11.2009 --- \textsl{Obrazki rekurencyjne i gramatyki} [\textsl{Recursive images and gramatics}] (XXVII SMS' Session \textsl{Mathematics in pictures}, Szczyrk)
  \item 04.12.2009 --- \textsl{Całe mnóstwo sposobów na narysowanie trójkąta Sierpińskiego} [\textsl{Lots and lots of ways to draw a Sierpiński triangle}] (talk for high schools)
  \item 17.12.2009 --- \textsl{Paradoks malarzy} [\textsl{Painter's Paradox}] (Scientific festival, Tychy)
  \item 01.02.2010 --- \textsl{Drawings of polynomial roots} (\nth{6} international Students' Conference on analysis, Síkfőkút)
  \item 12.03.2010 --- \textsl{Co potrafią narysować wielomiany?} [\textsl{What can be drawn by polynomials?}] (The $\pi$ day)
  \item 09.04.2010 --- \textsl{O pewnym równaniu funkcyjnym} [\textsl{A certain functional equation}] (talk for high schools)
  \item 29.04.2010 --- \textsl{O pewnej technice rysowania podzbiorów $\mathbb{R}^3$} [\textsl{A certain method of drawing the subsets of $\mathbb{R}^3$}] (talk for members of the SMS)
  \item 01.05.2010 --- \textsl{O pewnym twierdzeniu z teorii liczb} [\textsl{A certain theorem from the number theory}] (XXVIII SMS' Session \textsl{Alternative proofs}, Szczyrk)
  \item 07.05.2010 --- \textsl{Kilka słów o automatach skończenie stanowych} [\textsl{A few words about finite state machines}] (talk for high schools)
  \item 12.05.2010 --- \textsl{Od abstrakcji do zastosowań --- odległość Hamminga} [\textsl{From abstraction to applications --- Hamming distance}] (VI Powiatowy Drużynowy Konkurs Matematyczny o Puchar Dyrektora II LO im. M. Kopernika w Cieszynie)
  \item 20.07.2010 --- \textsl{Miara Lebesgue'a i zbiory niemierzalne} [\textsl{Lebesgue measure and non-measurable sets}] (\nth{7} SMS' Summer Camp \textsl{Measure Theory}, Zakopane)
  \item 23.07.2010 --- \textsl{Funkcje o wahaniu skończonym} [\textsl{Functions with finite fluctuation}] (\nth{7} SMS' Summer Camp \textsl{Measure Theory}, Zakopane)
  \item 27.08.2010 --- \textsl{O sitach} [\textsl{About sieves}] (Summer Maths Workshop 2010, Toruń)
  \item 30.10.2010 --- \textsl{O liczbie $\pi$} [\textsl{About $\pi$}] (talk for Unikids Jaworzno)
  \item 13.11.2010 --- \textsl{Matematyczna tresura żółwia} [\textsl{Mathematical turtle training}] (talk for Unikids Jaworzno)
  \item 19.11.2010 --- \textsl{Matematyka na niebie} [\textsl{Mathematics on the sky}] (talk for high schools)
  \item 27.11.2010 --- \textsl{Rozwinięcie $\pi$, superstringi i programowanie genetyczne} [\textsl{Representation of $\pi$, superstrings and genetic programming}] (XXIX SMS' Session \textsl{Mathematics and Computing Science}, Szczyrk)
  \item 10.12.2010 --- \textsl{O nieskończonościach} [\textsl{About infinities}] (Scientific festival, Tychy)
  \item 18.12.2010 --- \textsl{Gdyby świat miał ułamkowy wymiar} [\textsl{If world would have fractional dimension}] (talk for Unikids Jaworzno)
  \item 25.02.2011 --- \textsl{Wokół twierdzenia Riemanna} [\textsl{Around the Riemann series theorem}] (talk for high schools)
  \item 14.03.2011 --- \textsl{O pewnej szalonej funkcji} [\textsl{About some crazy function}] (The $\pi$ day)
  \item 02.05.2011 --- \textsl{Funkcja trzynastkowa} [\textsl{The thirteen function}] (XXX SMS' Session \textsl{Pathologies and paradoxes in mathematics}, Szczyrk)
  \item 27.05.2011 --- \textsl{Równania różniczkowe, czyli strzałki na płaszczyźnie} [\textsl{Differential equations, i.e. arrows on the plane}] (talk for high schools)
  \item 02.07.2011 --- \textsl{Rachunek operatorów Mikusińskiego} [\textsl{Mikusiński's operational calculus}] (\nth{8} SMS' Summer Camp \textsl{Applications of differential equations}, Zakopane)
  \item 13.09.2011 --- \textsl{Polish Jura --- not only for climbing} (ACAMiMM 2011, Bydgoszcz)
  \item 23.09.2011 --- \textsl{Ilu ludzi żyło na świecie?} [\textsl{How many people lived in the world?}] (Silesian Scientists' Night 2011, Katowice)
  \item 28.10.2011 --- \textsl{O pewnej programistycznej sztuczce} [\textsl{On a certain programming trick}] (talk for members of the SMS)
  \item 05.11.2011 --- \textsl{Szereg harmoniczny i cegły} [\textsl{Harmonic series and bricks}] (Aktywny w szkole -- aktywny w życiu, Katowice)
  \item 12.11.2011 --- \textsl{Światy możliwe} [\textsl{Possible worlds}] (XXXI SMS' Session \textsl{Motivations, intuitions, constructions}, Szczyrk)
  \item 02.12.2011 --- \textsl{Osobliwości z przedziału $[0, 1]$} [\textsl{Peculiarities of the $[0,1]$ interval}] (talk for high schools)
  \item 09.12.2011 --- \textsl{O maszynach Turinga} [\textsl{About Turing machines}] (Scientific festival, Tychy)
  \item 31.01.2012 --- \textsl{Few words about attractors} (\nth{8} international Students' Conference on analysis, Síkfőkút)
  \item 15.03.2012 --- \textsl{Jak natura radzi sobie z problemami?} [\textsl{How nature solves problems?}] (The $\pi$ day)
  \item 10.05.2012 --- \textsl{Attractors for semigroups in metric spaces} (talk for members of the SMS)
  \item 11.05.2012 --- \textsl{Implementacja świata w matematyce} [\textsl{Implementation of the world in mathematics}] (talk for high schools)
  \item 02.06.2012 --- \textsl{Nie mam pojęcia, czyli metody numeryczne dla równań różniczkowych} [\textsl{I do not know, i.e. the numerical methods for differential equations}] (XXXII SMS' Session \textsl{Algorithms}, Szczyrk)
  \item 12.07.2012 --- \textsl{Parada kłamców --- algorytm metropolis} [\textsl{Parade of liars ---  Metropolis algorithm}] (\nth{9} SMS' Summer Camp \textsl{Applications of probability theory}, Zakopane)
  \item 24.11.2012 --- \textsl{Złośliwe argumenty NWD} [\textsl{Mischievous arguments of GCD}] (XXXIII SMS' Session \textsl{Number theory and cryptography}, Ustroń, talk presented via the Internet)
  \item 03.12.2012 --- \textsl{Podstawy programu POV-Ray} [\textsl{Basics of the POV-Ray}] (talk for chair of the Mathematical Methods in Mathematics)
  \item 13.12.2012 --- \textsl{O pewnej konsekwencji własności Darboux} [\textsl{On some consequence of Darboux property}] (Scientific festival, Tychy)
  \item 08.01.2013 --- \textsl{O pewnym uogólnieniu równania Steinhausa} [About some generalization of Steinhaus functional equation] (1291 meeting of chair of functional equations, Katowice)
  \item 24.01.2013 --- \textsl{O istnieniu rozwiązania dla problemu optymalnych połączeń fotonicznych} [On the existence of solution of the problem of optimal photonic wire bonds] (171 meeting of Chair of the Mathematical Methods in Economy and Finances)
  \item 03.02.2013 --- \textsl{Some remarks on weakly Picard operators and their application in functional equations} (\nth{9} international Students' Conference on analysis, Ustroń)
  \item 15.03.2013 --- \textsl{O rzeczach niemożliwych} [\textsl{About impossible things}] (The $\pi$ day)
  \item 25.03.2013 --- \textsl{Pewna charakteryzacja trójkąta równobocznego} [\textsl{A characterization of an equilateral triangle}] (talk for members of the SMS)
  \item 07.06.2013 --- \textsl{Problem Dydony} [\textsl{Dido's problem}] (talk for members of the SMS)
  \item 04.10.2013 --- \textsl{Równania Maxwella i światłowody} [\textsl{Maxwell's equations and optical waveguides}] (talk for members of the SMS)
  \item 27.01.2014 --- \textsl{About applications of bifurcation theory to some problems arising from non-linear Maxwell's equations} (GRK1294 seminar)
  \item 11.04.2014 --- \textsl{Matematyka i łodzie (oraz okręty) podwodne} [\textsl{Mathematics and submarines}] (talk for high schools)
  \item 25.08.2014 --- \textsl{Calderón-Zygmund inequality} (GRK1294 workshop, Bad Herrenalb) (talk was given also on 14.08.2014 at the seminar of the Workgroup \textit{Nonlinear Partial Differential Equations})
  \item 23.09.2014 --- \textsl{Twierdzenie Crandalla-Rabinowitza i jego zastosowanie} [\textsl{Crandall-Rabinowitz theorem and its application}] (talk for members of the SMS)
  \item 02.02.2015 --- \textsl{Travelling waves for a nonlinear Maxwell's equations} (GRK1294 seminar)
  \item 27.03.2015 --- \textsl{O pewnym złośliwym ciągu całek} [\textsl{On some catty sequence of integrals}] (talk for members of the SMS)
  \item 27.03.2015 --- \textsl{Gra w stosy} [\textsl{Game of piles}] (talk for high schools)
  \item 07.07.2015 --- \textsl{Periodic solutions of some nonlinear wave equation} (seminar of the Workgroup \textit{Nonlinear Partial Differential Equations})
  \item 15.01.2016 --- \textsl{Existence of travelling waves for certain quasilinear/semilinear wave equations} (CRC 1173: Workshop of Projects A5 \& A6 Nonlinear Maxwell Equations, Karlsruhe)
  \item 08.04.2016 --- \textsl{Existence of some travelling waves for nonlinear Maxwell equations} (\nth{1} annual meeting of the CRC 1173, Rastatt)
  \item 21.12.2016 --- \textsl{O pewnej funkcji nigdzie wypukłej z wypukłym ograniczeniem górnym} [\textsl{About a certain nowhere convex function with a convex upper bound}] (talk for members of the SMS)
  \item 23.05.2017 --- \textsl{On the existence of travelling waves for some semi- and quasilinear wave equations} (International Conference on Elliptic and Parabolic Problems, minisymposium: \textit{PDEs arising in nonlinear optics}, Gaeta)
\end{itemize}
\noindent Object Oriented Programming Workshop for Students' Mathematical Society of the University of Silesia:
\begin{itemize}
  \item 18.11.2011
  \item 09.12.2011
  \item 12.01.2012
  \item 29.03.2012
\end{itemize}
